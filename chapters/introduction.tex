\section{Introduction} \label{introduction}
In this thesis we created an extension to the KEMS called KenticoApp. It allows clients to administrate their site from their smartphones. KenticoApp consists of two parts: the custom web API which stores and retrieves data from and to the database and the mobile app which allows the user to communicate with the system. 

The custom web API was created using the .Net framework. It uses KAPI calls and is called by the mobile app. For user authentication we decided to use access tokens (AT). ATs are leveraged to secure the communication between a user and the system. After signing in the user is given a random generated AT by the system and the system stores it in it's database. Before every API call the system checks the users AT against the database. For the call to be executed the AT has to exist in the database with the corresponding user ID and must not be expired. If this is not the case the user is redirected to the welcome page, where he has to sign in. To store them in the database we utilized the Entity Framework and implemented data access layer (DAL). For the purpose of this thesis we decided to represent the ATs as an entity using the Entity Framework. The entity contains the user identification (ID) a unique pseudo-random code and an expiration date and time. The code is of the type string and is generated with the pseudo-random generator Random(). Right after generating the code, it is tested against the database if a AT with the same code exists. If yes another code is generated and tested. If there is not the token is assigned the user ID and the date and time 10 minutes from the assignement. 

For the implementation of the mobile app we leveraged the Apache Cordva framework (ACF). The reason being that it is simple to use and supports seven platforms. As opposed to the Xamarin framework (XF) supporting three. Even though XF should be faster than ACF the difference between execution times of today's devices is negligible. The development was divided into two stages. For the appereance we decided to use JQuery Mobile. It is an HTML5-based user interface system which allows users to create aesthetically pleasing mobile elements by utilizing the languages Cascading Style Sheets (CSS) and HyperText Markup Language (HTML). As for the functionality we used the JQuery library which has a small learning curve and offers a fast way to add or delete elements or their behaviour. 

Chapter one introduces KEMS, web API and hybrid mobile applications. In the second chapter we describe the application architecture and the implementation of the extension of the KEMS in more detail. Finally, we valorise the achieved result and suggest other potential extensions or solutions.