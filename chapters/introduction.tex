\label{introduction}
KEMS is a~content management system (CMS) which allows clients to~form and manage their~web-sites using a~single user interface (UI). In this~thesis we created a~mobile app called KenticoApp which calls an~API that we also developed by extending the~API of~KEMS. An~API is a~collection of~functionality which a~programmer is able to~utilise in a~third party app. The~KenticoApp makes it~possible for clients to manage their~site from their~smartphones. It~consists of two parts: the~CAPI backend, which stores and retrieves data from and to the database, and the~mobile client app, which allows the~user to communicate with the~system.  The~functionality is divided into three main categories. The~first category represents the~system tasks such as restarting the~server, cleaning unused memory or cache and reading the~eventlog or general system information. The~second one operates with the~users and their~roles. It offers the~editing of the~user's first and last name and adding or removing their~roles. The~third and last category makes it possible to create or delete roles and edit them by~adding or removing permissions. To be able to perform all of the above actions the user has to be authenticated and authorized first. The authentication credentials are checked against the KEMS database using KAPI. Only global administrators are authorized.

The~backend was implemented in C\# .NET and communicates with the~KAPI. The mobile client app is a~Cordova app written in JavaScript, HyperText Markup Language (HTML) and Cascading Style Sheets (CSS). The~communication is ensured by asynchronous JavaScript and Extensible Markup Language (Ajax) in the~format JavaScript Object Notation (JSON). For the~purpose of~version control and backup we decided to use a~technology called Git. Our~Git project was hosted on the~web-based Git repository hosting service called GitHub. GitHub is an industry standard for hosting open-source software source code. 

Chapter one introduces KEMS, web API and hybrid mobile applications. In the second chapter we describe the application architecture and the implementation of the extension of the KEMS in more detail. Finally, we valorise the achieved result and suggest other potential extensions or solutions.
