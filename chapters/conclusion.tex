In this thesis we created a mobile app for the administration of KEMS called KenticoApp. To do so we had first to create a web API to operate the KAPI features. KenticoApp leverages our web API to communicate with the KAPI. 

One of the difficulties which occurred during the fulfilling of this project was for example the more complicated integration of ASP.NET Web API 2.0 within KEMS \cite{kenticoWebAPI}. KEMS utilises ASP.NET Web API 1.0 which makes this version more convenient to leverage rather than the newer one. Another complication took place while using the library JQM which simplifies the creating of aesthetically pleasing elements. On one hand it greatly simplifies the development of the UI, on the other hand it also sets limitations when trying to style elements in a way not supported by JQM out of the box. The tasks such as positioning buttons above the footer which would be easy using raw CSS were more cumbersome when JQM was used in the application because we had to override styles added by the framework.

\section{Evaluation}
Our app was user-tested by five people, all former informatics students. Because KEMS's clients are mostly from the informatics sphere, the choice of our testers seemed favorable. These individuals were explained what KEMS is and what it is used for. Then they got acquainted with KenticoApp on their own and they completed four given tasks. After that they had to fill a system usability scale questionnaire (SUS) \cite{sus}, which is made of 10 items and a short feedback. The items where about the effort made when using the app. Each person assigned a number from one to ten to each verdict depending on the agreement where one meant strongly disagree and five strongly agree. After evaluating the answers we concluded the app to be easy to use even without any tutorial but the visual appearance is not very appealing and the functionality was not integrated as intuitively. The test subjects missed a more colourful scheme and buttons. Another feature the testers missed was a better representation of data, e.g. the logs in the eventlog or roles divided into smaller groups. An issue was also missing feedback, particularly when clearing cache or unused memory. A characteristic which was appreciated was the simplicity of KenticoApp and visual consistency.

\section{Future Work}
\begin{description}
\item[CAPI]The CAPI is currently not RESTful. Following all conventions of this architecture would be an improvement which could be achieved in the future. Another advancement would be to secure the AT more. Currently the code generated as its ID is only pseudo random, therefore an attacker could determine the code of a specific AT and misuse it. 
\item[KenticoApp] As for KenticoApp, an enhancement would be to give the user an option to select which sites will be administrated. Also an option for sending a forgotten password or the ability to change it for the current user could be available. A polished UI would be an additional upgrade of the app. Access control would be a convenient trait. This would allow different users to manage different functionality without tempering with resources of each other.
For example, the confirmation and creating buttons could be green and the canceling and deleting buttons red. And the user feedback we collected during testing, such as better representation of data and missing feedback, could be implemented in the future.
\end{description}
